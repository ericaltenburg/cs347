\documentclass[preprint,11pt,3p]{article}

\usepackage{tocloft}
\usepackage{color}
\usepackage{hyperref}
\usepackage{graphicx}


\renewcommand{\cftsecleader}{\cftdotfill{\cftdotsep}}
\renewcommand{\abstractname}{Executive Summary}
\renewcommand{\labelenumii}{\theenumii}
\renewcommand{\theenumii}{\theenumi.\arabic{enumii}.}

\title{Cruise Control Software Development Version 0.04}

\author{
Eric Altenburg, Michael McCreesh, Hamzah Nizami, Constance Xu}
	
\date{
	Team Mike\endgraf\bigskip
	Stevens Institute of Technology\endgraf\bigskip
	CS 347 — Software Development Process\endgraf\bigskip
	\textit{We pledge our honor that we have abided by the Stevens Honor System.}}

\begin{document}

\maketitle
\newpage

\tableofcontents
\newpage

\begin{abstract}
	Team Mike is a startup initiative aimed at solving problems that come to light as
society begins to adopt new technologies. One of which pertains to autonomous
driving and “smart cars” as they are beginning to break into the automobile
market more and more every year with examples such as Tesla and Ford. In a
perfect world, if every driving car were to be a smart car with “autopilot,” then
it would make sense for each of them to communicate cruise control data with
each other to reduce traffic build-up and make traveling more efficient. Through
rotational leadership on a monthly basis, each team member possesses a set of
distinct skills that can translate to highly efficient work sessions. Aside from
developing a traditional cruise control, the aim is to make the software open source as it will serve as the foundation to a more interconnected logistical future in which autonomous cars can fully achieve
their potential within a growing technologically advanced society. 
\end{abstract}

\section {Introduction}
The past century has seen an explosion of innovation on a scale never before seen
in human history. The rapid development and refinement of a myriad of motor
technologies catalyzed the innovation process by allowing ideas to transfer at
rates never before seen. With the constraint of distance loosening thanks to
every iteration of the automobile, people have been granted more freedom to
do what they please. While the automobile has enhanced society in several
ways, there are some glaring problems that need to be addressed to continue
the current rate of human progress and innovation. One of the biggest problems
is driver fatigue, which is responsible for approximately 72,000 crashes annually.
Programming the automobile to be reliably autonomous to a degree is one way
to circumvent the issue of driver fatigue and making roads safer. That’s exactly
what cruise control aims to do. By moderating the speed of the vehicle by itself,
cruise controls aim to lessen the effects of driver fatigue on the road. However,
the technology is not perfect, and Team Mike aims to resolve that by creating
the perfect cruise control system. \par
When it was implemented in 1958, cruise control was suitable for that era,
however, as technology continually improves, the embedded software must as
well. For longer car trips, it does not make sense for users to constantly accelerate
and decelerate for several miles; this is where cruise control comes in. By being
able to set the speed, the user is now able to accurately set and maintain the
speed they wish to without having to constantly intervene which, over time,
would cause less harm to the vehicle as variable speed and RPMs are not as
desirable as that of near-constant. \par
Our group intends to use an Agile method to implement our version of cruise
control. This would encourage code sprints in two-week lengths where work
will be divided into smaller sections and distributed based on each members’
strengths. The Agile method may vary as to which one we specifically choose
but regardless, this project will not come to fruition using Waterfall or other
methods. The way that we are going to organize our group is through weekly
meetings and ensuring that everyone knows their tasks for the week. These
weekly meetings can also serve as a place where team members mention any
obstacles that have come in their way when trying to resolve a problem, and
also serve as a time where the team as a collective can try to think of solutions
around those obstacles. We also plan on giving real-time updates about any
progress made through Slack in order to ensure all members of the team are
equipped with the most recent developments of the project. We also intend on
using Java or C++ to implement our cruise control system. We believe that the
high-performance Java provides and the fact that its part of the Object-Oriented
Programming (OOP) paradigm makes it perfect for embedded systems. \par
There are a plethora of features that make up cruise control. As most cruise
controls are seen on vehicles, there are generally a few buttons: increase speed
(and sometimes decrease speed) and the button that starts cruise control. Once
cruise control is set, the user then has the ability to increase, decrease, or
maintain the speed of a vehicle. Furthermore, if the user presses on the brake,
then cruise control is automatically deactivated. The system also allows for the
operator of a vehicle to manually deactivate it. \par
To successfully implement the features of maintaining a steady speed and a
safe distance away from other cars, there are certain requirements to fulfill. At a
high level, proper hardware with fast, reliable sensors is critical for this project.
Furthermore, lives are fundamentally dependent on this product and therefore
it requires software that is error-resistant and thus a strong development and
QA (quality assurance) team. In addition, the software must also perform it’s
expected tasks of moderating speed, maintaining a safe distance and switching
back control to the driver when prompted to quickly. Having a cruise control
software that is slow would be ineffective, so speed and accuracy are critical
requirements for the software as well as the hardware. This is a mission-critical
system because if it were to fail, it would put the lives of people in danger.

\newpage
\section{Requirements}

\subsection{Input}
\begin{enumerate}
	\item The power button that allows for the state of the system to change from on to off or vice versa.
	\item When pressed, a button will either accelerate or decelerate in 1 mph increments. 
	\item When the brake is pressed, the cruise control system will unset the speed and give control back to the user until the user specifies a new speed to be set.
	\item If the entire car turns off, then a signal will be sent to turn off the cruise control system.
	\item RPM (Rotations Per Minute) sensor should be connected to the front axle and able to take readings for accurate speed calculations. 
	% \item RADAR sensor must be able to detect nearby vehicles/obstacles to avoid collision. %Do we really wanna do this
	\item Engine sensor must be able to take input from the engine to tell the cruise control system to turn on or off.
		\begin{enumerate}
			\item If the engine is turned on, the cruise control system must be ready for use within 3 seconds of the engine being turned on.
		\end{enumerate}
	\item If the gas pedal is pressed, the vehicle will continue to accelerate at the control of the user, but when the gas pedal is released the cruise control system will continue to the previously set speed.
	% \item Brake pedal sensor must be able to take input from user to stop when pressed and keep going when released.
\end{enumerate}

\subsection{Output}
\begin{enumerate}
	\item Keep a log of activity in the system files to help debug in the event of a malfunction.
	\item For every speed increase made by the user in the cruise control system, a visual indication by the software is necessary. This is so that the user minimizes their own human error. So, every time you increase the speed, you can see the cruise control speed on the car menu increasing by however much the user wants it to.
	\item The system displays successful activation or deactivation in 10 milliseconds. 
		\begin{enumerate}
			\item Numbers are subject to change depending on how inputs are received from surrounding system, but are expected to be in around that same ballpark.
		\end{enumerate}
	\item Keep a log of every time the cruise control system is used and at what speed written to the system files.
	% \item This module must be able to do what the user wants within a fast time frame, less than ten milliseconds. *Numbers are subject to change depending on how inputs are received from surrounding system, but are expected to be in around that same ballpark.*
	\item Once the cruise control system is turned on, it must be readily available. The engine sensor must be able to understand that the engine is on and tell the cruise control system that, if the user so wishes, it must activate. 
	\item With all the inputs it is taking from all the sensors, the software must be able to deliver the desired output for each function in less than 15 milliseconds. 
		\begin{enumerate}
			\item The brake pedal sensor must be able to tell the software that the user has stopped, and stop the cruise control system. *Numbers are subject to change depending on how inputs are received from surrounding system, but are expected to be in around that same ballpark.*
		\end{enumerate}
\end{enumerate}

\subsection{Functional}
\begin{enumerate}
	\item While the cruise control system is on and a speed is set, it must be able to increase and decrease the speed by 1 mph.
	\item Able to switch the cruise control system on or off provided the engine is on.
	\item Interpret engine on and off signal to allow for the cruise control system to be turned on within 3 seconds of received the signal.
	\item Only allow the cruise control system to be activated when at a minimum speed of 25 mph. 
		\begin{enumerate}
			\item If the speed is set to 25 mph, the user cannot decrease the speed below 25 mph.
		\end{enumerate}%What if we set speed and and set speed below this
	\item Maximum speed that cruise control system can be set to is at 125 mph.
	\item While the cruise control system is on, if a user is driving at a speed of at least 25 mph and decides to set the speed, the cruise control system will maintain that speed.
	\item If the user presses and continues to press the brakes, the user will not be able to set the speed.
	\item If the user presses and continues to press the gas pedal, the user can set the current speed for the cruise control system, but it will immediately pause as stated in the input requirements then resume after the user releases the gas pedal.
\end{enumerate}

\subsection{Security}
\begin{enumerate}
	\item No external interface to reduce potential tampering. This applies to both the hardware such as sensors and the general cruise control system software. There would be no easy access to the cruise control system hardware or software so that the likelihood of someone being able to create issues is reduced substantially. 
	\item No Internet or Bluetooth connection. This is so that no bad actors can meddle with the car's system and hence, keeps the drivers safer as cyber security becomes more of an issue.
	\item An administrator will need hardware to make changes to the mechanisms. The administrator must be authorized to make these changes and they must be a trusted third party or those who created the cruise control system themselves. 
\end{enumerate}

\section{Requirements Analysis Model}

\subsection{UML Use Cases \& Diagram} 
\begin{enumerate}
	\item Use Case 1: User activates cruise control system
		\begin{enumerate}
			\item User requests an activation of the cruise control system.
			\item Cruise control system provides a visual feedback that it is ready for activation.
			\item Cruise control system activates.
		\end{enumerate}
	\item Use Case 2: User sets the speed
		\begin{enumerate}
			\item Cruise control system requests values from sensors.
			\item Sensors provide values for approval to set cruise control system at current speed.
			\item Cruise control system requests the Engine Management System (EMS) set the speed at current position.
			\item EMS Speed (Throttle) is set at current speed.
			\item Cruise control system provides visual feedback to the user that the cruise control is set and working.
			\item Sensors provide the changing environmental information to the cruise control unit (such as speed, request for increase/decrease speed, and brake).
			\item Cruise control system detects the changes from the sensors and request adjusting speed or deactivating Cruise Control system accordingly.
			\item Speed (Throttle) position is continuously set to new values to ensure that the speed remains constant.
			\item Speed is continuously reported to the cruise control system.
		\end{enumerate}
	\item Use Case 3: User adjusts speed
		\begin{enumerate}
			\item User requests an adjustment of cruise control system speed.
			\item Cruise control system provides visual feedback that the system will alter the speed of the vehicle.
			\item Cruise control system slowly adjusts the speed of the vehicle to match that of the request.
			\item When the desired speed is reached, the cruise control system will provide visual feedback that the adjustment has been completed.
		\end{enumerate}
	\item Use Case 4: User deactivates cruise control system
		\begin{enumerate}
			\item User requests a deactivation of the cruise control system.
			\item Cruise control provides visual feedback that it is ready for deactivation.
			\item Cruise control system deactivates.
		\end{enumerate}
	\item See Figure ~\ref{fig:ccUML1} for general use cases of cruise control system.
		\begin{figure}[h!]
			\includegraphics[width=0.9\textwidth]{images/useCaseUML.png}
			\caption{Sample UML diagram for cruise control system use cases.}
			\label{fig:ccUML1}
		\end{figure}
\end{enumerate}

\subsection{UML Class-Based Modeling}

\subsection{UML CRC Model Index Card}

\subsection{UML Activity Diagram}

\subsection{UML Sequence Diagram}
\begin{enumerate}
	\item 	See Figure ~\ref{fig:ccUML2} for sequence diagram of cruise control system.\begin{figure}[h!]
				\includegraphics[width=0.9\textwidth]{images/ccUML.png}
				\caption{Sample UML sequel diagram for cruise control system functionality.}
				\label{fig:ccUML2}
			\end{figure}
\end{enumerate}
	

\subsection{UML State Diagram}

\end{document}

