%=======================02-713 LaTeX template, following the 15-210 template==================
%
% You don't need to use LaTeX or this template, but you must turn your homework in as
% a typeset PDF somehow.
%
% How to use:
%    1. Update your information in section "A" below
%    2. Write your answers in section "B" below. Precede answers for all 
%       parts of a question with the command "\question{n}{desc}" where n is
%       the question number and "desc" is a short, one-line description of 
%       the problem. There is no need to restate the problem.
%    3. If a question has multiple parts, precede the answer to part x with the
%       command "\part{x}".
%    4. If a problem asks you to design an algorithm, use the commands
%       \algorithm, \correctness, \runtime to precede your discussion of the 
%       description of the algorithm, its correctness, and its running time, respectively.
%    5. You can include graphics by using the command \includegraphics{FILENAME}
%    6. If you want to use code, use \begin{lstlisting} after making modifications to the lstset.
\documentclass[11pt]{article}
\usepackage{amsmath,amssymb,amsthm}
\usepackage{graphicx}
\usepackage[margin=1in]{geometry}
\usepackage{fancyhdr}
\usepackage{listings}
\usepackage{color}
\setlength{\parindent}{0pt}
\setlength{\parskip}{5pt plus 1pt}
\setlength{\headheight}{13.6pt}
\newcommand\question[2]{\vspace{.25in}\hrule\textbf{#1: #2}\vspace{.5em}\hrule\vspace{.10in}}
\renewcommand\part[1]{\vspace{.10in}\textbf{(#1)}\par}
\newcommand\algorithm{\vspace{.10in}\textbf{Algorithm: }}
\newcommand\correctness{\vspace{.10in}\textbf{Correctness: }}
\newcommand\runtime{\vspace{.10in}\textbf{Running time: }}
\definecolor{javared}{rgb}{0.6,0,0} % for strings
\definecolor{javagreen}{rgb}{0.25,0.5,0.35} % comments
\definecolor{javapurple}{rgb}{0.5,0,0.35} % keywords
\definecolor{javadocblue}{rgb}{0.25,0.35,0.75} % javadoc
\lstset{language=C,
basicstyle=\ttfamily,
keywordstyle=\color{javapurple}\bfseries,
stringstyle=\color{javared},
commentstyle=\color{javagreen},
morecomment=[s][\color{javadocblue}]{/**}{*/},
numbers=left,
numberstyle=\tiny\color{black},
stepnumber=1,
numbersep=10pt,
tabsize=4,
showspaces=false,
showstringspaces=false}
\pagestyle{fancyplain}
\lhead{\textbf{\NAME}}
\chead{\textbf{{\COURSE} HW\HWNUM}}
\rhead{\today}
\begin{document}
%Section A==============Change the values below to match your information==================
\newcommand\NAME{Eric Altenburg}  % your name
\newcommand\COURSE{CS 347}
\newcommand\HWNUM{1}              % the homework number
%Section B==============Put your answers to the questions below here=======================

% no need to restate the problem --- the graders know which problem is which,
% but replacing "The First Problem" with a short phrase will help you remember
% which problem this is when you read over your homeworks to study.

\begin{center}
	\textit{\textbf{Pledge:} I pledge my honor that I have abided by the Stevens Honor System.} - \textbf{\NAME}
\end{center}


\question{1.6}{As software becomes more pervasive, risks to the public (due to faulty programs) become an increasingly significant concern. Develop a doomsday by realistic scenario in which the failure of a computer program could do great harm, either economic or human.}


\question{2.8}{Is it possible to combine process models? If so, provide an example.}
	Yes, it is possible to combine process models, in fact, some software development departments/companies do not use a traditional process model. Instead, they end up using a proprietary process model which is just a combination of other more traditional process models that better fit their work flow/products.\par

	A few examples of combined process models include:
	\begin{enumerate}
		\item Evolutionary process model
			\begin{itemize}
				\item 
			\end{itemize}
	\end{enumerate}


\question{2.9}{What are the advantages and disadvantages to developing software in which quality is "good enough"? That is, what happens when we emphasize development speed over product quality.}


\question{3.2}{Describe agility (for software projects) in your own words.}


\question{5.1}{Based on your personal observations of people who are excellent software developers, name three personality traits that appear to be common among them.}


\question{6.6}{Of the eight core principles that guide process (discussed in Section 6.1.1), what do you believe is more important?}


\question{7.1}{Why is it that many software developers don't pay enough attention to requirements engineering? Are there ever circumstances where you can skip it?}


\question{7.5a}{Develop a complete use case for making a withdrawal from an ATM.}


\question{8.1}{Is it possible to begin coding immediately after a requirements model has been created? Explain your answer, and then argue the counterpoint.}


\question{8.10}{How does a sequence diagram differ from a state diagram? How are they similar?}


\end{document}